\section{References to the Manual of Indulgences}
The following norms and grants which are relevant to the given prayers are reproduced from \citetitle[the][]{manual-indulgences}:
\subsection{Relevant Norms}
\newcommand{\norm}[1]{\textbf{N#1.}\newline}
\newcommand{\subnorm}[1]{\sectionsign{}#1.}
\norm{1}
An indulgence is a remission before God of the temporal punishment for sins, whose guilt is forgiven, which a properly disposed member of the Christian faithful obtains under certain and clearly defined conditions through the intervention of the Church, which, as the minister of Redemption, dispenses and applies authoritatively the treasury of the expiatory works of Christ and the saints.

\norm{3}
The faithful can obtain partial or plenary indulgences for themselves, or they can apply them to the dead by way of suffrage.

\norm{17}
\subnorm{1} In order to be capable of gaining indulgences one must be baptized, not excommunicated, and in the state of grace at least at the completion of the prescribed works.

\subnorm{2} To gain an indulgence, one must have at least the general intention of doing so and must carry out the enjoined works at the stated time and in due fashion, according to the sense of the grant.

\norm{18}
\subnorm{1} A plenary indulgence can be acquired only once in the course of a day; a partial indulgence can be acquired multiple times.

\subnorm{2} The faithful however can obtain the plenary indulgence at the hour of death, even if they have already gained one on the same day.

\norm{20}
\subnorm{1} To gain a plenary indulgence, in addition to excluding all attachment to sin, even venial sin, it is necessary to perform the indulgenced work and fulfill the following three conditions:
sacramental confession, Eucharistic Communion, and prayer for the intention of the Sovereign Pontiff.

\subnorm{2} A single sacramental confession suffices for gaining several plenary indulgences;
but Holy Communion must be received and prayer for the intention of the Holy Father must be recited for the gaining of each plenary indulgence.

\subnorm{3} The three conditions may be fulfilled several days before or after the performance of the prescribed work;
it is, however, fitting that Communion be received and the prayer for the intention of the Holy Father be said on the same day the work is performed.

\subnorm{4} If the full disposition is lacking, or if the work and the three prescribed conditions are not fulfilled, saving the provisions given in Norm 24 and in Norm 25 regarding those who are ``impeded,'' the indulgence will only be partial.

\subnorm{5} The condition of praying for the intention of the Holy Father is fully satisfied by reciting one Our Father and one Hail Mary;
nevertheless, one has the option of reciting any other prayer according to individual piety and devotion, if recited for this intention.

\norm{22} An indulgence attached to a prayer can be acquired by reciting the prayer in any language, provided that the translation is approved by the competent ecclesiastical authority.

\norm{23} To gain an indulgence it is sufficient to recite the prayer alternately with a companion or to follow it mentally while it is being recited by another.

\subsection{Grants for Prayers}
% Consider indenting subgrants and subsubgrants using a method described at:
% https://tex.stackexchange.com/questions/35933/indenting-a-whole-paragraph
% package enumitem might be useful here too
\newcommand{\grant}[2]{\subsubsection*{#1 #2}}
\newcommand{\subgrant}[1]{\sectionsign{}#1.}
\newcommand{\subsubgrant}[1]{#1\degreesign{}}
\hypertarget{grant2}{\grant{2}{Act of Dedication of the Human Race to Jesus Christ the King}}
A plenary indulgence is granted to the faithful who on the solemnity of Our Lord Jesus Christ, King of the Universe, publicly recite the act of dedication of the human race to Christ the King (\foreign{Iesu dulcissime, Redemptor});
a partial indulgence is granted for its use in other circumstances.

\hypertarget{grant3}{\grant{3}{Act of Reparation}}
A plenary indulgence is granted to the faithful who, on the solemnity of the Most Sacred Heart of Jesus, publicly recite the act of reparation (\foreign{Iesu dulcissime}); a partial indulgence is granted for its use in other circumstances.

\hypertarget{grant7}{\grant{7}{Eucharistic Adoration and Procession}}
\subgrant{1} A plenary indulgence is granted to the faithful who \subsubgrant{2} piously recite the verses of the \foreign{Tantum ergo} after the Mass of the Lord's Supper on Holy Thursday during the solemn reposition of the Most Blessed Sacrament.

\subgrant{2} A partial indulgence is granted to the faithful who \subsubgrant{2} offer any duly approved prayer to Jesus present in the Blessed Sacrament (e.g., the \foreign{Adoro te devote}, the prayer \foreign{O sacrum convivium}, or the \foreign{Tantum ergo}).

\hypertarget{grant8}{\grant{8}{Eucharistic and Spiritual Communion}}

\subgrant{1} A plenary indulgence is granted to the faithful who \subsubgrant{2} on any of the Fridays of Lent devoutly recite after Communion the prayer \foreign{En ego, O bone et dulcissime Iesu} before a crucifix.

\subgrant{2} A partial indulgence is granted to the faithful who, using any duly approved pious formula, make

\subsubgrant{1} an act of spiritual communion;

\subsubgrant{2} an act of thanksgiving after Communion (e.g., \foreign{Anima Christi}; \foreign{En ego, O bone et dulcissime Iesu}).

\hypertarget{grant9}{\grant{9}{Examination of Conscience and Act of Contrition}}
A partial indulgence is granted to the faithful who, especially in preparation for sacramental confession,

\subgrant{1} examine their conscience with the purpose of amendment;

\subgrant{2} devoutly recite an act of contrition, according to any legitimate formula (e.g., the \foreign{Confiteor}, the psalm \foreign{De profundis}, or the psalm \foreign{Miserere}, or any of the gradual or penitential psalms).

\hypertarget{grant11}{\grant{11}{Week of Prayer for Christian Unity}}
\subgrant{2} A partial indulgence is granted to the faithful who devoutly recite an appropriately approved prayer for the unity of Christians (e.g. \foreign{Omnipotens et misericors Deus}).

\grant{15}{Mental Prayer}
A partial indulgence is granted to the faithful who for their personal edification devoutly spend time in mental prayer.

\hypertarget{grant17}{\grant{17}{Prayers to the Blessed Virgin Mary}}
\subgrant{1} A plenary indulgence is granted to the faithful who

\subsubgrant{1} devoutly recite the Marian rosary in a church or oratory, or in a family, a religious community, or an association of the faithful, and in general when several of the faithful gather for some honest purpose;

\subsubgrant{2} devoutly join in the recitation of the rosary while it is being recited by the Supreme Pontiff and broadcast live by radio or television.

In other circumstances, the indulgence will be partial.

The rosary is a prayer formula consisting of fifteen decades of Hail Marys preceded by the Our Father, during the recitation of which we piously meditate on the corresponding mysteries of our redemption.
Regarding the plenary indulgence for the recitation of the Marian rosary, the following is prescribed:
\begin{enumerate}
	\item The recitation of a third part of the rosary is sufficient, but the five decades must be recited without interruption.
	\item Devout meditation on the mysteries is to be added to the vocal prayer.
	\item In its public recitation the mysteries must be announced in accord with approved local custom, but in its private recitation it is sufficient for the faithful simply to join meditation on the mysteries to the vocal prayer.
\end{enumerate}

\subgrant{2} A partial indulgence is granted to the faithful who

\subsubgrant{1} devoutly recite the canticle of the \foreign{Magnificat};

\subsubgrant{2} either at dawn, noon, or evening devoutly recite the \foreign{Angelus} with its accompanying versicles and prayer or, during the Easter season, the \foreign{Regina caeli} antiphon with its usual prayer;

\subsubgrant{3} devoutly address the Blessed Virgin Mary with some approved prayer (e.g., \foreign{Maria, Mater gratiae}; the \foreign{Memorare}; the \foreign{Salve Regina}; the \foreign{Sancta Maria, succurre miseris}; or the \foreign{Sub tuum praesidium}.)

\hypertarget{grant18}{\grant{18}{Prayers to One's Guardian Angel}}
A partial indulgence is granted to the faithful who devoutly invoke the care of their guardian Angel with a duly approved prayer (e.g. \foreign{Angele Dei}).

\hypertarget{grant19}{\grant{19}{Prayers in Honor of St. Joseph}}
A partial indulgence is granted to the faithful who invoke St. Joseph, spouse of the Blessed Virgin Mary, with a duly approved prayer (e.g., \foreign{Ad te, beate Ioseph}).

\hypertarget{grant20}{\grant{20}{Prayers in Honor the Apostles Peter and Paul}}
A partial indulgence is granted to the faithful who devoutly recite the prayer \foreign{Sancti Apostoli Petre et Paule}.

\hypertarget{grant22}{\grant{22}{Novenas, Litanies, and the Little Offices}}
A partial indulgence is granted to the faithful who

\subsubgrant{1} devoutly assist at public novenas (e.g., prior to the Solemnities of the Nativity of the Lord, or of Pentecost, or of the Immaculate Conception of the Blessed Virgin Mary);

\subsubgrant{2} devoutly recite approved litanies (e.g., of the Most Holy Name of Jesus, the Most Sacred Heart of Jesus, the Most Precious Blood of Our Lord Jesus Christ, the Blessed Virgin Mary, St. Joseph, and of the Saints);

\subsubgrant{3} piously recite an approved little office (e.g., of the Passion of Our Lord Jesus Christ, the Most Sacred Heart of Jesus, the Blessed Virgin Mary, the Immaculate Conception, or St. Joseph).

\hypertarget{grant23}{\grant{23}{Prayers of the Eastern Churches}}
\subgrant{2} A partial indulgence is granted to the faithful who, in accordance with particular times and circumstances, devoutly recite one of the following prayers: a Prayer of Thanksgiving (from the Armenian tradition); Evening Prayer, Prayer for the Faithful Departed (from the Byzantine Tradition); the Prayer of the Shrine, the Prayer \foreign{``Lakhu Mara''} known as To You, O Lord, (from the Chaldean Tradition); a Prayer for the Offering of Incense, Prayer to Glorify Mary, the Mother of God (from the Coptic Tradition); Prayer for the Remission of Sins, Prayer for Following in the Footsteps of Christ (from the Ethiopian Tradition); Prayer for the Church, Prayer After the Celebration of the Liturgy (from the Maronite Tradition), and the Intercessions for the Faithful Departed from the Liturgy of St. James (from the Syro-Antiochian Tradition).

\hypertarget{grant24}{\grant{24}{Prayers for Benefactors}}
A partial indulgence is granted to the faithful who, moved by supernatural gratitude, devoutly recite a duly approved prayer for benefactors (e.g., \foreign{Retribuere dignare, Domine}).

\hypertarget{grant26}{\grant{26}{Prayers of Supplication and Acts of Thanksgiving}}
\subgrant{1} A plenary indulgence is granted to the faithful who devoutly assist either at the recitation or solemn singing of

\subsubgrant{1} the \foreign{Veni Creator}, either on the first day of the year to implore divine assistance for the course of the whole year, or on the solemnity of Pentecost;

\subsubgrant{2} the \foreign{Te Deum}, on the final day of the year, to offer thanks to God for gifts received throughout the course of the entire year.

\subgrant{2} A partial indulgence is granted to the faithful who,

\subsubgrant{1} at the beginning and the end of the day,

\subsubgrant{2} in starting and completing their work,

\subsubgrant{3} before and after meals,

devoutly offer some legitimately approved prayer of supplication and act of thanksgiving (e.g., \foreign{Actiones nostras}; \foreign{Adsumus}; \foreign{Agimus Tibi gratias}; \foreign{Benedic, Domine}; \foreign{Domine, Deus Omnipotens}; \foreign{Exaudi nos}; the \foreign{Te Deum}; the \foreign{Veni Creator}; the \foreign{Veni Sancte Spiritus}; \foreign{Visita, quaesumus, Domine}).

\hypertarget{grant28}{\grant{28}{Profession of Faith and Acts of the Theological Virtues}}
\subgrant{2} A partial indulgence is granted to the faithful who

\subsubgrant{1} renew their baptismal vows in any formula;

\subsubgrant{2} devoutly sign themselves with the sign of the cross, using the customary words: ``In the Name of the Father and of the Son and of the Holy Spirit. Amen'';

\subsubgrant{3} devoutly recite either the Apostles' Creed or the Niceno-Constantinopolitan Creed;

\subsubgrant{4} recite an Act of Faith, Hope, and Charity in any legitimate formula.

\hypertarget{grant29}{\grant{29}{For the Faithful Departed}}
\subgrant{1} A plenary indulgence, applicable only to the souls in purgatory, is granted to the faithful who,

\subsubgrant{1} on any and each day from November 1 to 8, devoutly visit a cemetery and pray, if only mentally, for the departed;

\subsubgrant{2} on All Souls' Day (or, according to the judgment of the ordinary, on the Sunday preceding or following it, or on the solemnity of All Saints), devoutly visit a church or an oratory and recite an Our Father and the Creed.

\subgrant{2} A partial indulgence, applicable only to the souls in purgatory, is granted to the faithful who

\subsubgrant{1} devoutly visit a cemetery and at least mentally pray for the dead;

\subsubgrant{2} devoutly recite lauds or vespers from the Office of the Dead or the prayer ``Eternal rest''.

\newpage